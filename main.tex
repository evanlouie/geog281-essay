\documentclass[man,donotrepeattitle,letter]{apa6}

\usepackage[american]{babel}
\usepackage{csquotes}
\usepackage[style=apa,sortcites=true,sorting=nyt,backend=biber]{biblatex}
\usepackage{float}
\floatstyle{boxed}
\restylefloat{figure}
\usepackage{graphicx}
\usepackage{lipsum}
% \usepackage[utf8]{inputenc}
% \DeclareUnicodeCharacter{00A0}{ }
\DeclareLanguageMapping{american}{american-apa}
\addbibresource{main.bib}


\title{GEOG281 Essay Assignment - Response to The Economist's \textit{The Pacific}}
\shorttitle{GEOG281 Essay Assignment}

\author{Evan Louie}

\affiliation{University of British Columbia}

\abstract{The Economist's special report, \textit{The Pacific}, effectively outlines the raising economic importance that the relations within the Pacific play.  Paying special attention to the actions of Asiatic superpower, China, The Economist paints a picture in which the future of all players within the Pacific Rim are heavily interconnected and dependant upon one another: with the fate of one affecting the fate of all.}

\keywords{GEOG281, essay, assignment, Pacific Rim, Pacific-Asia, east Asia, south-east Asia, Japan, China, Korea, United States, relations, formation, integration, economist}

\authornote{Intended for GEOG281 affiliated staff}
\begin{document}
\maketitle

\tableofcontents
\newpage
\section{Preamble}
This paper is in response to Henry Tricks' report, The Pacific.  While comprised of nine smaller articles, this paper will respond to Trick's report as a whole and address the overall narrative and concerns posited by Tricks.

\section{Main issues of concern}
In The Economist's special report, \textit{The Pacific}, author Henry Tricks goes in depth into the role that the pacific plays in the global economy and the future the reemerging economic superpower, China, will have in it.  Often ignored or played down as a single solitary unit within the global economy, the pacific is becoming evermore connected and is the most dynamic region within the global economy.  Often thought of as just the relationship between the U.S and China, regional links between the smaller countries within Southeast Asia, China, Japan, and across the pacific to the U.S and Latin America, where Chile's largest trading partner is China (formerly the U.S), has fostered economic growth in both sides of the pacific and outlines the growing importance of the interconnections within the Pacific Rim and its regional players.

\begin{quotation}
  ``The Mediterranean is the ocean of the past, the Atlantic is the ocean of the present and the Pacific is the ocean of the future'' -- John Hay, U.S Secretary of State 1898-1905
\end{quotation}

Sharing a common culture, largely a common language, and, most noticeably, common institutions, not least, the North Atlantic Trade Organization (NATO), the Atlantic has been the core economic ocean in bridging the West for the past several centuries.  The Pacific carries none of these traits and is what leads to the tensions existing within the Pacific today, countering the economic promise that it brings. With a superpower rivalry between the U.S and China taking place across the Pacific, especially the western Pacific, China is actively testing the rules established by the U.S after World War II and the rules in which the global economy has been run, kept peace, and fostered trade for the past 70 years. China, without directly breaking the rules, is testing their limits; with territorial claims in the South China Sea and Japan's Senkaku Islands, the challenge for the next few decades in the Pacific is to create institutions akin to NATO that can diffuse such tensions (Tricks, 2014e).

Spouting a national GDP of 9,181,204 (millions of US\$), China's economic performance is second only to the United States and the regional conglomerate, the European Union. Although one might find it apt to replace Japan with China in Akamatsu's third ``flying geese'' paradigm, the paradigm it is no longer apt in today's Pacific economy, with it more resembling a tangled web than the iconic v-formation. While China maintains itself as an economic superpower, it is integrated into a ``web of economic interdependence'' in which destabilizing could lead to disastrous results (Tricks, 2014d).

Overall, Tricks paints a rather grim and cautionary image of China and its role within the global economy.  With ever-rising tensions between economic powers such as China and Japan, as well as the smaller countries in South East Asia, interconnectivity between these nations still continues to grow as economic growth incentivizes them to.  However, as global powers such as the U.S, China, and Japan continue to have tensions grow, their dependency on one another will continue to increase, as the interconnected economic web of the Pacific connects them together.  The core concern raised seems to not revolve around the uncertainty the economy faces under the globalizing economy, but what can/will occur due to the China's aberrant actions.


\section{Issues left out}
Once World War II's Pacific Asian Theatre drew to a close, the Pacific underwent rapid changes in development and structure. The U.S quickly began rebuilding and reorganizing Japan's physical and social infrastructure, leading to Japan being of the U.S's main source of political and military influence in East Asia. Subsequently, the Korean War enabled the U.S to gain another political outpost in the East Asian economic theatre.  The outcomes of these wars have led to heavy U.S involvement in East Asian affairs and the U.S having an incredibly strong political and economic standing within the region. Even now, Sokolsky et al.'s (2000) \textit{The Role of Southeast Asia in US Strategy Toward China} denotes how the U.S's primary goal in Southeast Asia is to prevent the U.S from being denied economic, political, and military access to the Pacific.  So the validity of the narrative and concerns that Tricks alludes to are muddied as these points are left out, outlining China as the exclusive active party trying to establish itself as the primary superpower of the Pacific Rim.

Historically, China was one the worlds foremost economic powers.  Before the Opium Wars, China was a self-contained nation that did little to no importing and was capable of exclusively relying on trading their products along the Silk Road for British silver.  Following the Opium Wars, China was forced to open to the West and internal wars and uprisings quickly lead to the dissolving of the Qing Dynasty, the subsequent empowerment of the Republic of China (Taiwan), and after WWII, the rise of the People's Republic of China (POC).  Since the West's intrusion into Chinese affairs, China's government became chaotic and instability rose until the communist party, the POC, took power. Similar occurrences happened in Japan that also lead to political unrest. Although decidedly a different age than when these events occurred, it stands to question whether or not the tactics used by present day China are in fact very different than those done by the U.S to maintain the political and economic status it created a century ago. Albeit China's actions are more directly noticeable than those of the U.S, China is attempting to carve out a sociopolitical and economic zone akin to the one the U.S was able to form when the standards of global political play had not yet been set.

When looking at the overall criticisms and concerns raised by Tricks, it becomes quite apparent that the overall argument is somewhat Western-centric.  From the point of view of the economically powerful West, the actions of China look almost imperialistic in nature.  However, when looking at the actions of the U.S, even in the present, it becomes clear that the U.S is also attempting a sort of imperialistic push into the Pacific, albeit in a subtler manner. With Khalilzad et al.'s (1999) \textit{The United States and a rising China: Strategic and Military Implications} denoting their economic, political, and militaristic goals in Southeast Asia and Kerrey's (2001) \textit{The United States and Southeast Asia: A Policy Agenda for the New Administration} also noting the U.S's goal of ``prevention of intraregional conflict and domination by an outside power or coalition''.  The U.S fears the economic and political power China wields and the possible formation of an economic co-prosperity sphere in which the U.S will not be able to play a part (i.e. RCEP vs. TPP) (Tricks, 2014a).



\section{Reasons for leaving out issues}
The Economist is a publication that attempts to look at events from as relatively an objective point of view an economic outlook can offer:  Attempting to view the world through the looking glass of modern day free-market capitalistic economics.  So it stands that any report, whether it be regarding North America, Pacific Asia, the Eurozone, or any other location would, and should, fall under the assumption that all players within a report follow the same standards and rules which have governed the global economy since the end of World War II.  As such, the report, while not explicitly stating as such, implies the actions of China as aberrant and counter to the standards in which the global economy currently runs.  Whether or not it is a fair argument that China, or any other nation, has the right to disobey the rules, which they themselves never agreed to or had any hand in the forging of is a common point of contention (Tricks, 2014i).  Arguments based on cultural/ethical relativism and social contract theory will offer differing opinions, with the former often being concluded to be an unworkable theory and the latter holding more merit. But again, this argument is stemmed in ethical philosophy and is outside the journalistic scope of a publication such as The Economist.

Economic theory is built on the premise that all players act rationally and predictably.  With China attempting actions such as expansions into the South China Sea and Senkaku Islands and indirect boycotts of Japanese products (Tricks, 2014g), it is to be expected that economists would look at China as both a land offering a promising future but with possibilities for turbulence. In terms of the actions of the U.S, they, as a global economic player, act within the constraints set in place after WWII.  Choosing less direct methods to achieve a presence in East Asia, the U.S will always appear as a more economically viable candidate than that of China.

The Economists main goal is to explain the world in terms of free-market economics. As such, it makes sense that there would a political pull towards a more Western-centric point-of-view, as the current global economic setting was set in motion by it and it is the leading center of such a style of market.  Commenting on the possibility of China acting belligerently if concerned about its economic future, Tricks makes little comment on what action the U.S would take if it or China were to lose economic footing. Alluding to the mutually assured destruction the U.S and China will undergo if either of them undergoes economic turmoil, the distances the U.S government would be willing to go to preserve \textit{Pax Americana} is extremely understated.

By fulfilling its goals in showing the world through a relatively objective economic lens, The Economist falls short on addressing several issues in its evaluation of the Pacific and its local superpower, China.  Not limited to the historical relations found within and without the Pacific, the report underplays the role the U.S has and had in creating and maintaining the sociopolitical and economic setting which it currently wishes to play a part. With the goal of Pax Americana strictly at the forefront of 21\textsuperscript{st} century U.S political thought, the political pull (i.e. free-market economics) that The Economists subscribes to inadvertently, and possibly intentionally, portrays the U.S an overly positive light.

\section{Rewritten: change in political orientation?}
If rewritten to include increased historical context of how Pacific Rim formation occurred or the explicit stating of the U.S's current political goals in the Pacific and how it plans to achieve them, the overall ``political'' orientation of the report would change drastically.  In its current form, the report focuses on the Pacific Rim's present and future in the global free-market economy.  As such, additions of historical context and U.S's goal to maintain Pax Americana would not only shift the orientation from that of a solely economic/business one, to one more akin to a historical recounting and geopolitics, showing less alignment with the free-market economic paradigm which the West and The Economist subscribe.

While Tricks understandably does not wish for an economic report to get bogged down by the minute details of history, the report does display his understanding and appreciation for how the relations in the Pacific did not only start in the 19\textsuperscript{th} century, but as far back as four centuries ago when Japanese samurai, Hasekura Tsunenaga, attempted to open trade with New Spain (Mexico) (Tricks, 2014a).  However, taking a more historical approach in increasing the understanding of how Pacific Rim formation occurred, specifically that which occurred from the late 19th and thereafter, could very well aid in offering the reader a better understanding of how and why relations are in their current state.

With China's relatively quick rise to economic power, it comes to question whether or not history is beginning to repeat itself.  China is now not only one of the largest producers of material goods, but is also quickly taking strides into the realm of digital intellectual properties and e-commerce, with companies such as Alibaba and Tencent surpassing their North American counterparts (Tricks, 2014g).  With the IMF already reporting China to have overtaken the U.S economy (Bird, 2014), China's growth is understandably alarming to Western powers, looking akin to its former self in the early 19\textsuperscript{th} century.  This of course would run contrary to the positings made by Tricks, believing China's economic growth to be unsustainable and notes China's comparatively low average GDP per person; positing that the legitimacy of the Chinese government hangs on the ability to improve the standards of living.

To include notions of the present day geopolitical doings of the U.S, as highlighted by Kerrey and Khalilzad et al.'s reports, would run somewhat counter to the narrative and political affiliation set by The Economist and Tricks.  Throughout the report, China is often portrayed as the sole provocateur in terms of political/economic status in East and Southeast Asia.  However, as history shows, the U.S and the West have played major roles in East Asia's formation and continues to do so to this day. This is not to imply that The Economist nor Tricks have any kind of underlying political affiliation, but to say that the way in which The Economist views the world and events, while apt for particular uses, is narrowed by the innate usage of free-market economics as its lens and constrains the ability to see the overall larger picture. That being, although touted as the bastion of free-market capitalism, the means in which the U.S goes about positioning itself within East Asia is just as politically charged as China's. The Economist, which focuses on global economics, misses this fact and thus appears to politically align with the U.S and the West.  To do away with such alignment would change the political orientation of the report from one stemmed in global economics to one of geopolitics.




\nocite{*}
\printbibliography

\newpage
[This page is intentionally left blank.]

\end{document}

%
% Please see the package documentation for more information
% on the APA6 document class:
%
% http://www.ctan.org/pkg/apa6
%
